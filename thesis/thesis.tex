% !TeX encoding = UTF-8
% !TeX program = pdflatex
% !TeX spellcheck = en_US

\documentclass[LaM,binding=0.6cm]{sapthesis}

\usepackage{microtype}

\usepackage{hyperref}
\hypersetup{pdftitle={Usage example of the Sapthesis class for a Laurea Magistrale thesis in English},pdfauthor={Francesco Biccari}}

% Remove in a normal thesis
\usepackage{lipsum}
\usepackage{curve2e}
\definecolor{gray}{gray}{0.4}
\newcommand{\bs}{\textbackslash}

% Commands for the titlepage
\title{of the Sapthesis class\\ for a Laurea Magistrale thesis in English}
\author{Francesco Biccari}
\IDnumber{1234567}
\course{Fisica}
\courseorganizer{Facolt\`{a} di Scienze Matematiche, Fisiche e Naturali}
\AcademicYear{2012/2013}
\copyyear{2013}
\advisor{Prof. Nome Cognome}
\advisor{Dr. Nome Cognome}
\coadvisor{Dr. Nome Cognome}
\authoremail{biccari@email.com}

\examdate{16 April 2013}
\examiner{Prof. Nome Cognome}
\examiner{Prof. Nome Cognome}
\examiner{Dr. Nome Cognome}
\versiondate{\today}



\begin{document}

\frontmatter

\maketitle

\dedication{Dedicated to\\ Donald Knuth}

\begin{abstract}
This document is an example which shows the main features of
the \LaTeXe\ class \texttt{sapthesis.cls} developed by Francesco Biccari
with the help of GuIT (Gruppo Utilizzatori Italiani di \TeX).
\end{abstract}

\begin{acknowledgments}
Ho deciso di scrivere i ringraziamenti in italiano
per dimostrare la mia gratitudine verso i membri
del GuIT, il Gruppo Utilizzatori Italiani di \TeX, e, in particolare,
verso il prof. Enrico Gregorio.
\end{acknowledgments}

\tableofcontents






\mainmatter

\chapter{Introduction}
\section{General idea}
\section{Problem statement}
\section{Previous work}
\section{State of the art}

\chapter{Development}

\section{Dataset}
\section{Dataset}
\section{Dataset}

\chapter{Conclusion}
\section{Real world applications}





\backmatter
% bibliography
%\cleardoublepage
%\phantomsection
%\bibliographystyle{sapthesis} % BibTeX style
%\bibliography{bibliography} % BibTeX database without .bib extension

\end{document}
